\documentclass[10pt, conference, compsocconf]{IEEEtran}
\hyphenation{op-tical net-works semi-conduc-tor}

\usepackage{cite}
\usepackage{epsfig}
\usepackage{graphicx}
\usepackage{subfigure}

\begin{document}
\bibliographystyle{IEEEtran}

\title{Online Training Machine Learning based Decision Maker for Mobile
Offloading Framework}

\author{\IEEEauthorblockN{Heungsik Eom, Renato Figueiredo}
\IEEEauthorblockA{Advanced Computing and Information Systems Laboratory\\
Electrical and Computer Engineering\\
University of Florida, Gainesville, Florida, USA\\
\{hseom, renato\}@acis.ufl.edu}
\and
\IEEEauthorblockN{Huaqian Cai, Gang Huang}
\IEEEauthorblockA{Operating System and Middleware Laboratory\\
School of Electronics Engineering and Computer Science\\
Peking University, Beijing, China\\
\{caihq12, hg\}@pku.edu.cn}
}

\maketitle

\begin{abstract}
%
%OpenCL has emerged as the open standard for parallel programming for
%heterogeneous platforms enabling a uniform framework to discover,
%program, and distribute parallel workloads to the diverse set of compute
%units in the hardware.
%
%For that reason, there have been efforts exploring the advantages of
%parallelism from the OpenCL framework by offloading GPGPU workloads
%within an HPC cluster environment.
%
%In this paper, we present an OpenCL-based remote offloading framework
%designed for mobile platforms by shifting the motivation and
%advantages of using the OpenCL framework for the HPC cluster environment
%into mobile cloud computing where OpenCL workloads can be exported from
%a mobile node to the cloud.
%
%Furthermore, our offloading framework handles service discovery, access
%control, and data privacy by building the framework on top of a social
%peer-to-peer virtual private network, SocialVPN. 
%
%We developed a prototype implementation and deployed it into local- and
%wide-area environments to evaluate the performance improvement and
%energy implications of the proposed offloading framework.
%
%Our results show that, depending on the complexity of the workload and
%the amount of data transfer, the proposed architecture can achieve more
%energy efficient performance by offloading than executing locally.
\end{abstract}

\begin{IEEEkeywords}
Mobile platform, computation offloading, machine learning, runtime
decision maker, online training
\end{IEEEkeywords}

\section{Introduction}
%
Over the last decade, remote offloading techniques have emerged as
intelligent means to overcome the constraints of the limited resources
from mobile platforms, smartphones and tabletPCs, so that these types of
devices delegate computationally intensive computing tasks to more
powerful external resources such as personal workstations or cloud
servers.
%
Initially, most of research interests on remote offloading techniques
have focused on core mechanisms in which \textit{what to offload} and
\textit{how to offload} have been primarily considered. 
%
The research community has studied various approaches to implement mobile
offloading framework such as application partitioning~\cite{spectra,
maui, cuckoo}, thread migration~\cite{clonecloud, comet}, and
application migration~\cite{hung}.\\
%
\indent However, benefits from offloading computation-intensive portions
of mobile applications can be influenced by various internal and
external factors such as application requirements, network conditions,
and computing capabilities of mobile or external devices.
%
Thus, \textit{whether to offload or execute locally} needs to be
decided periodically by monitoring aforementioned dynamic features on
runtime.
%
Otherwise, incorrect offloading decisions may cause the performance
degradation or worse energy consumption.
%
For that reason, research focuses have been naturally shifted into
dynamic scheduling or decision making problems for mobile offloading
framework.
%
For example, Kwon et al.~\cite{kwon} consider a simple rule-based
decision maker in which the framework decides to offload the mobile
computation only when the data transfer size is greater than a certain
threshold.
%
MAUI~\cite{maui} utilizes a linear regression model among predefined
features to make offloading decisions.\\
%
%Also our previous paper suggested to apply machine learning
%techniques to the adaptive scheduling problem for mobile offloading
%framework by considering various machine learning algorithms~\cite{ml}.
%
\indent Even though these studies on the decision maker for mobile
offloading systems take dynamic features such as data transfer size or
network conditions to make offloading decisions, it is impractical for
these approaches to build a globally well-defined offloading decision
policy while considering all possible cases against dynamic mobile
environments.
%
Furthermore, it is difficult to generalize the above efforts for various
mobile application use case scenario, since different applications need
to have different offloading decision policies due to different
application requirements or characteristics.
%
Therefore, in practice, it is necessary for the decision maker to learn
from self-observation of the previous decision correctness and to
dynamically adapt the decision policy on runtime, thereby an identical
decision model can be \textit{generally} applied to various mobile
applications without any predefined decision policies.\\
%
\indent In this paper, we aim to develop a general framework for a
runtime adaptive decision maker for mobile offloading framework by
employing various types of machine learning techniques.
%
By applying machine learning techniques to scheduling problems
for mobile offloading framework, the machine learning classifier can
make decisions on whether the mobile computations should be offloaded to
external resources or executed in local.
%
To this end, we modularized our previous work on the machine learning
based runtime decision maker~\cite{ml} in which any appropriate machine
learning classifiers can be employed for the offloading decision maker.
%
Also, this can be plugged and played with any types of mobile offloading
frameworks in conjunction with the well-defined APIs to monitor and
acquire dynamic features 


which can be used for any mobile
offloading frameworks
and applied it to Java-based mobile
offloading system called \textit{Dpartner}~\cite{dpartner}.
%

%
We evaluated the cost and performance for three machine learning
algorithms, instance based learning, perceptron, and na\"{i}ve bayes with
respect to classifier build time, classification time, and decision
accuracy.\\
%
\indent After investigating the cost and performance of three machine 
learning algorithms for the runtime offloading decision maker, we
describe how we enable the online training mechanism for the runtime
decision maker.
%
The framework supports an online training mechanism for the
machine learning based runtime decision maker such that it feedbacks and 
learns from observation on the previous offloading decision
performance, and adapts its decision making policy on runtime.
%
Even though there have been prior related studies which suggest
utilizing machine learning techniques for mobile computing environments,
to the best of our knowledge, our work is the first to consider the
online training mechanism for the machine learning based decision maker
for mobile offloading framework.\\   
%
\indent The rest of the paper is organized  as follows.
%
In Section II, we overview previous studies on offloading decision
problems in mobile offloading frameworks, as well as the use of machine
learning techniques for various scheduling problems.
%
Section III discusses the challenges in online training machine learning
runtime scheduler for mobile offloading framework.
%
In Section IV and V, we explain and evaluate our implementation of the
online training ML scheduler.
%
Also, Section VI describes the current and potential applications for
our work.
%
Finally, we conclude the paper in Section VII.
%
\section{Related Works}
%
\section{Adaptive Decision Maker for Remote Offloading System}
%
\subsection{Offloading Performance}
%
\subsection{Machine Learning based Runtime Decision Maker}
%
\section{Challenge on Online Training ML based Runtime Decision Maker
for Mobile Offloading System}
%
\subsection{Offline vs. Online}
%
\subsection{Requirement for Online Training ML based Decision Maker for
Mobile Offloading System}
%
\section{Implementation of Online Training for ML based Decision Maker}
%
\subsection{Java-Based Mobile Offloading System}
%
\subsection{Online Training}
%
\section{Evaluation}
%
\subsection{Training Cost}
%
\subsection{Offloading Decision Performance}
%
\section{Use Case}
%
\section{Conclusion and Future Work}
% 
%\section*{Acknowledgement}
%This material is based upon work supported in part by the National Science
%Foundation under Grant No. 0910812, 0758596, 0855031, and 1265341.
%
%Any opinions, findings, and conclusions or recommendations expressed in
%this material are those of the author(s) and do not necessarily reflect
%the views of the National Science Foundation.
%
%\bibliographystyle{IEEEtran}
\bibliography{ipccc14}
\end{document}


